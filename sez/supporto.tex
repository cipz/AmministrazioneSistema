\chapter{Attività e risorse di supporto}\label{ch:supporto}

In questo capitolo vengono descritte le attività ed altre risorse considerate secondarie e di supporto all'implementazione del nuovo sistema informatico dell'\istituto.

Tali risorse possono essere di natura informatica, come hardware, software e altri materiali, o di natura aziendale come strutture, documenti, personale e requisiti di formazione.

\section{Documentazione}

	Per poter illustrare correttamente il nuovo sistema e l'implementazione dell'intero progetto, \azienda~produrrà documenti quali:
	\begin{itemize}[noitemsep]
		
		\item \textit{Composizione dei Processi Aziendali}: contiene l'analisi del vecchio sistema gerarchico, la transizione ai nuovi BPR e una descrizione dettagliata di questi;
		
		\item \textit{Analisi e architettura della rete e dei sistemi interni}: contiene un'analisi del vecchio sistema informatico a livello architetturale e di rete, la descrizione dettagliata della nuova rete, inclusiva dell'hardware utilizzato e delle postazioni dei dipendenti, del piano di indirizzamento IP e del partizionamento della nuova rete;

		\item \textit{Documento di avanzamento dei lavori}: questo documento conterrà tutti i rapporti redatti riguardanti l'avanzamento dei lavori;
		
		\item \textit{Documentazione relativa ai test}: tale documento conterrà tutte le informazioni necessarie per quanto riguarda i test (da \textit{unit test} a \textit{test di accettazione}) svolti sul sistema e che possono essere rifatti a cadenza temporale, oppure svolti per particolari eventi, come, per esempio, la release e l'implementazione di una nuova feature.
		Una porzione di questo documento può essere dedicata ai rischi, descrivendoli come specificato in \ref{sec:rischi}, a come evitarli o mitigarli e a quali test sono stati eseguiti per poterli prevenire;
		
		\newpage
		\item \textit{Documento aggiornato dei ``progetti innovativi''}: vista la richiesta dell'\istituto~di lavorare su progetti innovativi, è necessario tenere traccia di questi in un documento che sia in costante aggiornamento.
		Tale documento conterrà la descrizione di ciascun progetto attraverso l'uso di template precedentemente concordati dai tecnici di \azienda~e dal \proponente, in base alle necessità di quest'ultimo.
		
	\end{itemize}

	Tali documenti serviranno non solamente ai futuri tecnici dell'\istituto~per capire il funzionamento del sistema, ma anche agli stakeholders, che, tramite tali elaborati, potranno seguire l'avanzamento del progetto.

\section{Materiale informatico}

	Tutto il materiale informatico e relativa documentazione che l'\istituto~possiede dovrà essere messa a disposizione ai dipendenti di \azienda~per poter garantire che tutti i servizi attualmente erogati siano noti durante la transizione e che questi possano essere replicati nel nuovo sistema.

	\subsection{Hardware}
		
		Come specificato dal \proponente~nel capitolato, tutto l'hardware dell'attuale sistema sarà preso in carica dall'\offerente.
		Questo include tutta l'infrastruttura tecnologica specificata dall'\istituto~nel capitolato.
		
	\subsection{Software}
	
		Come per l'hardware, anche il software verrà preso in carica dal \offerente.
		I sistemi applicativi di tutte le aree verranno messe a disposizione ad \azienda, la quale si prenderà cura di mantenerli attivi durante la transizione e di effettuare backup periodici, completi o incrementali, in caso fosse necessario effettuare un \rollback.
	
\section{Strutture}

	Per la corretta installazione del nuovo sistema informatico, e, di conseguenza, il corretto completamento del progetto, è necessario l'accesso alle strutture ospedaliere.
	Questo implica una concessione periodica, valida dall'inizio del progetto, di credenziali e di permessi di accesso da parte del \proponente~ad \azienda~per poter permettere ai suoi dipendenti di entrare e uscire dagli edifici che compongono l'\istituto.

\section{Personale}

	Anche il personale messo a disposizione dall'\istituto~rappresenta un asset importante per il corretto svolgimento del progetto.
	
	\subsection{Requisiti di staff}
	
		Per implementare correttamente il progetto non sforando il tempo e il budget prefissati, \azienda~necessita che l'\istituto~metta a disposizione i propri dipendenti di varie aree.
		
		\azienda, per riuscire a creare una piattaforma ad hoc, necessita di avere feedback non solamente dai principali stakeholders, ma anche da coloro che la useranno giornalmente.
		Per questo inizialmente verranno somministrati alcuni questionari, non soltanto riguardo le feature più utilizzate, ma anche sull'interfaccia grafica più adatta all'utilizzo dei dipendenti.
		
		I tecnici dell'\istituto~verranno affiancati dai dipendenti di \azienda~per trovare la soluzione migliore ai bisogni presentati all'interno del capitolato.
	
	\subsection{Formazione dello staff}
	
		La formazione dello staff dell'\istituto~da parte di \azienda~è uno step principale per quanto riguarda la fase di riconsegna del progetto.
		Senza un corretto corso di formazione ed un sufficente affiancamento, il personale dell'\istituto~non sarà in grado di utilizzare correttamente i nuovi strumenti sviluppati appositamente da \azienda.
	
		\subsubsection{Formazione tecnica}
			
			Questa riguarda lo staff che sarà a stretto contatto con il nuovo sistema di \helpdesk~e che dovrà rispondere efficacemente alle richieste degli utenti, agendo da \textit{single point of contact}, come spiegato in \ref{sec:scopo_helpdesk}.
		
			Tale personale deve essere esperto in materia informatica, avendo preferibilmente una certificazione in \textit{ITIL Foundations}.
			La conoscenza dei processi ITIL permette ai dipendenti di poter capire a fondo come è strutturato il nuovo sistema e come poter intevenire professionalmente in caso di bisogno.
			
			I dipendenti sottoposti alla formazione tecnica dovranno avere una profonda conoscenza del processo di Business Process Reengineering, e verranno sottoposti a test di valutazione, sia in fase di presa in carico sia prima della conclusione del BPR, per verificare la corretta comprensione dei nuovi processi aziendali.
			
			Il personale continuerà ad essere formato anche dopo la riconsegna del progetto: verranno calendarizzate sessioni regolari di formazione per poter permettere ai dipendenti di imparare il nuovo sistema e ad essere pratici dei nuovi processi.
			
			Tali sessioni di aggiornamento possono essere tenute nel normale orario di lavoro, ma, per garantire una corretta e continuativa erogazione dei servizi di Service Desk, ne verranno proposte di più (due o tre) alle quali i dipendenti potranno aderire.
			
		\subsubsection{Formazione non tecnica}
			
			In aggiunta alla formazione del personale a stretto contatto con le tecnologie e con l'architettura del \helpdesk, vi è anche un training per i dipendenti dell'\istituto~che non fanno parte dei tecnici.
			
			A tale staff verrà effettuata una completa formazione sulla nuova interfaccia del sistema, sulle nuove funzionalità che sono state aggiunte e sui nuovi metodi per contattare il Service Desk in caso di bisogno.
			
			Oltre alla formazione iniziale, che verrà effettuata dai dipendenti di \azienda~presso la sede dell'\istituto, verranno forniti ai dipendenti tutti i documenti e i manuali, non tecnici, che sono stati redatti durante lo svolgimento del progetto e che possono essere di aiuto in caso di studio personale.
			
			Come per la formazione tecnica, anche per questo tipo di formazione è necessario un corso di aggiornamento, in questo specifico caso a cadenza minore.
			Le date e la frequenza di queste sessioni verranno concordate successivamente tra \azienda~e l'\istituto.
		